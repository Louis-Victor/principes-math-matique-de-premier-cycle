\documentclass{book}

\usepackage{amsmath}
\usepackage{amsfonts}
\usepackage[T1]{fontenc}
\usepackage[french]{babel}
\usepackage[autolanguage]{numprint} % for the \nombre command
\usepackage{hyphenat}
\hyphenation{mate-mática recu-perar}
\usepackage{amsthm}
\usepackage{enumitem}
\usepackage[hidelinks]{hyperref}
\hypersetup{
	colorlinks,
	citecolor=black,
	filecolor=black,
	linkcolor=black,
	urlcolor=black
}
\usepackage{import}

\newcommand{\powerset}{\mathcal{P}}
\newcommand{\ideal}{\vartriangleleft}

\theoremstyle{plain}
\newtheorem{thm}{Théorème}[chapter] % reset theorem numbering for each chapter
\newtheorem{lm}{Lemme}[thm]
\newtheorem{prop}{Proposition}[thm]
\newtheorem{cor}{Corollaire}[thm]

\theoremstyle{definition}
\newtheorem{defn}[thm]{Définition} % definition numbers are dependent on theorem numbers
\newtheorem{ex}[thm]{Example} % same for example numbers
\newtheorem{rmrk}[thm]{Remarque}


\title{Principes Mathématique de Premier Cycle}
\author{Louis-Victor}


\begin{document}
\maketitle
\date{}
\tableofcontents
\newpage
\part{Algèbre}
\label{part:alg}
\chapter{Algèbre Linéaire}
\label{ch:alg-lin}
\section{Polynomes Iréductibles}
\label{sec:alg:lin:ireduct}

\begin{defn}
Soit $K$ un corps et $p(x)\in K[x]$ un polynome. $p(x)$ est dit iréductible s'il est de degré positif et ne se factorise pas de manière non-triviale. C'est à dire que si $p(x)=f(x)g(x)$ pour certains $f(x),g(x)\in K[x]$ alors $\deg f=0$  ou $deg g=0$.
\end{defn}


\begin{lm}
\label{lm:alg:lin:ireduct:complex-roots}
Soit $p(x)\in\mathbb{R}[x]$ avec $\deg p\geq2$ et $\alpha\in\mathbb{C}$ une racine  de  $p(x)$. Alors $\overline{\alpha}$ est une racine de $p(x)$
\end{lm}
\begin{proof}
Posons $p(x)=a_0+a_1x+a_2x^2+...+a_nx^n$. On a donc

\begin{align*}
0&=p(\alpha)\\
0&=a_0+a_1\alpha+a_2\alpha^2+...+a_n\alpha^n\\
\overline{0}&=\overline{a_0+a_1\alpha+a_2\alpha^2+...+a_n\alpha^n}\\
0&=\overline{a_0+a_1\alpha+a_2\alpha^2+...+a_n\alpha^n}\\
&=\overline{a_0}+\overline{a_1\alpha}+\overline{a_2\alpha^2}+...+\overline{a_n\alpha^n}\\
&=a_0+a_1\overline{\alpha}+a_2\overline{\alpha^2}+...+a_n\overline{\alpha^n}\\
0&=p(\overline{\alpha})
\end{align*}

\end{proof}
\begin{prop}
Soient $p(x)\in\mathbb{C}[x]$ et $q(x)\in\mathbb{R}[x]$.

\begin{enumerate}
\item\label{prop:item:alg:lin:ireduct:clos} $p(x)$  est iréductible si et seulement si $\deg p=1$.
\item $q(x)$ est iréductible si et seulement si $\deg q=1$ ou $\deg q =2$ et $q(x)$ n'a aucune racine.
\end{enumerate}
Le point \ref{prop:item:alg:lin:ireduct:clos} s'applique également pour tout corps algébriquement clos.
\end{prop}
\begin{proof}
Soit $p(x)\in\mathbb{C}[x]$ un polynome iréductible.

Si $\deg p>2$ alors soit $p(x)$ a une racine réele $x_0$ ou $p(x)$ a une racine complexe $\alpha\in\mathbb{C}\setminus\mathbb{R}$.

Si $p(x_0)=0$ alors $p(x)=(x-x_0)q(x)+r(x)$. Mais comme $p(x_0)=0=(x_0-x_0)q(x_0)+r(x_0)=r(x_0)$ alors $r(x)=0$. Donc $p(x)=(x-x_0)q(x)$ avec $\deg q\geq0$. Donc $p(x)$ n'est pas iréductible.

Si $p(\alpha)=0$ alors $p(\overline{\alpha})=0$ par le lemme \ref{lm:alg:lin:ireduct:complex-roots}. On a donc $(x-\alpha)(x-\overline{\alpha})=x^2-(\alpha+\overline{\alpha})x+\alpha\overline{\alpha}$ qui divise $p(x)$ donc $p(x)$ n'est pas iréductible. $\Longrightarrow\Longleftarrow$
\end{proof}


\section{Théorème de Décomposition primaire}
\label{sec:alg:lin:dec-primaire}

\begin{lm}
\label{lm:alg:lin:dec-primaire:lema-1}
Soit $p(x)$ un polynome iréductible qui divise $f(x)g(x)$ avec $f(x),g(x)\in K[x]$. Alors $p(x)|f(x)$ ou $p(x)|g(x)$.
\end{lm}
\begin{proof}
Sans perte de généralité, supposons que $f(x)$ et $g(x)$ sont unitaires. Supposons que $p(x)\not|f(x)$. Soit $m(x)=pgcd(p(x),f(x))$.
\end{proof}
\begin{rmrk}
On peut prouver ce lemme avec le fait que $K[x]$ est un domaine a idéaux principaux (DIP) et que dans un DIP si un élément est iréductible alors celui ci est premier.
\end{rmrk}

\begin{thm}\textbf{Décomposition primaire.}
\label{th:alg:lin:dec-primaire:dec-primaire}
Soit $V$ un espace vectoriel de dimention arbitraire sur un corps $K$ et $T:V\longrightarrow V$ une transformation linéaire tel que $p(T)=0$ pour un certain polynome $p(x)=p_1^{a_1}(x)p_2^{a_2}(x)...p_r^{a_r}(x)\in K[x]$ où chaque $p_i(x)$ est iréductible et $p_i(x)\neq p_j(x)$ si $i\neq j$.

Alors $V=\ker\left(p_1^{a_1}(T)\right)\bigoplus...\bigoplus\ker\left(p_r^{a_r}(T)\right)$
\end{thm}

\chapter{Groupes}
\label{ch:alg:group}

\section{Groupes}
\label{sec:alg:group:group}
\section{Groupes Quotient}
\label{sec:alg:group:quotients}


\chapter{Anneaux}
\label{ch:alg:ring}
\input{algebra/ring.tex}

\end{document}
