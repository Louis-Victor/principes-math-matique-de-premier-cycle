\documentclass{book}

\usepackage{amsmath}
\usepackage{amsfonts}
\usepackage[T1]{fontenc}
\usepackage[french]{babel}
\usepackage[autolanguage]{numprint} % for the \nombre command
\usepackage{hyphenat}
\hyphenation{mate-mática recu-perar}
\usepackage{amsthm}
\usepackage{enumitem}
\usepackage[hidelinks]{hyperref}
\hypersetup{
	colorlinks,
	citecolor=black,
	filecolor=black,
	linkcolor=black,
	urlcolor=black
}
\usepackage{import}

\newcommand{\powerset}{\mathcal{P}}
\newcommand{\ideal}{\vartriangleleft}

\theoremstyle{plain}
\newtheorem{thm}{Théorème}[chapter] % reset theorem numbering for each chapter
\newtheorem{lm}{Lemme}[thm]
\newtheorem{cor}{Corollaire}[thm]

\theoremstyle{definition}
\newtheorem{defn}[thm]{Définition} % definition numbers are dependent on theorem numbers
\newtheorem{ex}[thm]{Example} % same for example numbers
\newtheorem{rmrk}[thm]{Remarque}


\title{Principes Mathématique de Premier Cycle}
\author{Louis-Victor}


\begin{document}
\maketitle
\date{}
\tableofcontents
\newpage
\part{Algèbre}
\chapter{Groupes}
\label{ch:group}

\section{Groupes}
\label{sec:group-group}
\section{Groupes Quotient}
\label{sec:group-quotients}

\label{part:alg}
\chapter{Anneaux}
\label{ch:ring}
\section{Introdction}
\label{sec:ring-intro}

\subsubsection{Prérequis}
Il est recomandé d'avoir lu et d'être familier avec les sections \ref{sec:group-group} à \ref{sec:group-quotients} du chapitre \ref{ch:group} sur les groupes.

\subsection{Histoire de la Théorie des Anneaux}
La théorie des anneaux est une discipline d'origine du $19^e$ siècle mais qui a fleuris vers les années 1930. Julius Wilhelm Richard Dedekind ($1831-1916$) était un mathématicien allemand qui a grandement dévéloppé la théorie des anneaux commutatif. Durant les premières décénies du $20^e$ siècle, Emmy Noether ($1882-1935$) explore les anneaux non-commutatif et les idéaux. D'autres noms importants incluent Nathan Jacobson (USA, $1910-1999$), Irving Kaplansky (Canada, $1917-2006$), Joseph wedderburn (UK, $1882-1948$) et plus encore.

La grande idée derrière la théorie des Anneaux est de généraliser les propriétés des entiers. Quel structures algébriques agissent de manières similaires? Ce qui est fait en imposant des restrictions sur les corps en permetant des éléments non-inversible et un produit qui n'est pas nécéssairement commutatif.

Certains exemples d'anneaux importants sont les entiers, les polynomes à plusieur variables sur les nombres complexes, les matrices avec des entrées réeles.

\subsection{Notions de Bases}
\begin{defn}
\label{def:ring}
Un anneau $A$ est un ensemble muni de deux opérations binaires: l'addition $+$ et la multiplication $\times$ ($a\times b$ sera noté $ab$) tel que
\begin{enumerate}
\item $(A,+)$ est un groupe abélien d'élément neutre $0$ ou $0_A$.
\item $\times$ est un opération associative, c'est à dire que $(ab)c=a(bc)$.
\item $\times$ distribue sur $+$, c'est à dire que $a(b+c)=ab+ac$ et $(b+c)a=ba+ca$.
\item Il existe un $1$ ou $1_A\in A$ tel que $1a=a1=a$ pour tout  $a\in A$. Cette propriété n'est pas nécéssaire pour un anneau.
\end{enumerate}

L'élément $1$ s'appel l'unité ou l'identité multiplicative de l'anneau.

Un anneau avec une identité multiplicative est dit unitaire et un anneau sans identié multiplicative est dit non-unitaire.

Pour le reste de cette section, nous allon considérer uniquement les anneaux unitaires.
\end{defn}

\begin{defn}
\label{def:ring-comut}
L'anneau $A$ est commutatif si les élément commutent sous $\times$. 
\end{defn}

\begin{rmrk}
On permet que $0_A=1_A$.
\end{rmrk}

\begin{ex}
$\mathbb{Z}$,$\mathbb{Z}_n$, $\mathbb{Z}[x]$ et $M_n\left(\mathbb{Z}\right)$ sont tous des anneaux sous l'addition et la multiplication habituelle.

$D=\left\{ f:\mathbb{R}\rightarrow\mathbb{R}\right\}$ est un anneau si on prend
\begin{align*}
+:D\times D&\rightarrow D\\
(f,g)&\mapsto f(x)+g(x)\\
\times:D\times D&\rightarrow D\\
(f,g)&\mapsto f(x)g(x)
\end{align*}

Par contre si on prend la composition comme operation multiplicative, alors $D$ ne sera pas un anneau.
\begin{align*}
f\circ(g+h)\neq f\circ g+f\circ h
\end{align*}

Si $A$ et $B$ sont des anneaux alors $A\times B$ est un anneau.

Similairement si $A_i$ est un anneau pour tout $1\leq i\leq n$ alors $A_1\times A_2\times...\times A_n$ est aussi un anneau.
\end{ex}

\begin{defn}
\label{def:ring-div0}
Soit $a\in A$ un anneau.

\begin{itemize}
\item $a$ est un diviseur de zéro à gauche s'il existe un $b\in A\setminus\{0\}$ tel que $ab=0$
\item $a$ est un diviseur de zéro à droite s'il existe un $b\in A\setminus\{0\}$ tel que $ba=0$
\item $a$ est un diviseur de zéros'il existe un $b\in A\setminus\{0\}$ tel que $ab=ba=0$
\end{itemize}

On dénote l'ensemble des diviseurs de zéro d'un anneau $A$ par $DZ(A)$.

Si $A$ n'a aucun diviseur de zéro alors on appel $A$ un anneau intègre.
\end{defn}

\begin{rmrk}
En anglais, un anneau intègre commutatif s'appel un \textit{Integral Domain}.
\end{rmrk}

\begin{defn}
Un sous-ensemble  $S$ d'un anneau $A$ est un sous-anneau de $A$ si $S$ lui même est un anneau sous les operations de $A$ et $1_S=1_A$.
\end{defn}

\begin{thm}
Un sous-ensemble $S$ d'un anneau $A$ est un sous-anneau $\iff$ $1_A,ab, a-b\in S$ pour tout $a,b\in S$.
\end{thm}

\section{Idéaux}
\label{sec:ideal}
\begin{rmrk}
Un sous-groupe $N$ d'un groupe $G$ est dit normal si $g\star n\in N$ pour tout $g\in G,n\in N$ où $g\star n$ dénote la  conjugaison $g\star n=gng^{-1}$.
\end{rmrk}

\begin{defn}
Un sous-ensemble non-vide $I$  d'un anneau $A$ est un idéal à gauche si $ax,x+y\in I$ pour tout $x,y\in I$ et $a\in A$.

Un sous-ensemble non-vide $I$  d'un anneau $A$ est un idéal à droite si $xa,x+y\in I$ pour tout $x,y\in I$ et $a\in A$.
\end{defn}

\begin{rmrk}
$I$ est un sous-groupe abélien mais pas toujours un sous-anneau. Surtout si $I$ est un sous-anneau alors $I=A$ car $1\in I$.

$1\in I\Longrightarrow a1\in I$ pour tout $a\in A\Longrightarrow a\in I \Longrightarrow A\subseteq I\Longrightarrow A=I$.
\end{rmrk}

\begin{defn}
Si $I$ est un idéal à gauche et à droite d'un anneau $A$ alors on dit que $I$ est un idéal bilatère ou simplement un idéal de $A$.

Dans ce cas, on écrit $I\ideal A$.
\end{defn}

\begin{defn}
On dit que $I$ est un idéal propre si $I\ideal A$ et $I\neq A$.
\end{defn}

\begin{defn}
Un idéal $I$ est dit non-trivial s'il n'est  pas le groupe trivial. C'est à dire que $I\neq\{0\}$.
\end{defn}

\begin{ex}
Soit $A$ un anneau commutatif. Pour tout $r\in A$ on peut obtenir l'idéal principal généré par $r$, qu'on note $\langle r\rangle$, tel que $\langle r\rangle:=\left\{ ar | a\in A\right\}$. L'idéal principal de $r$ est le plus petit idéal contenant $r$.

La démonstration que $\langle r\rangle$ est un idéal est laissé en exercice.

Similairement, $\langle r_1,r_2,...,r_n\rangle=\left\{ a_1r_1+a_2r_2+...+a_nr_n | a_i\in A\right\}$ est un idéal de $A$.
\end{ex}

\begin{ex}
Si $A=\mathbb{Z}$ et $n\in\mathbb{Z}$ alors $\langle n\rangle=n\mathbb{Z}=\left\{nm | m\in\mathbb{Z}\right\}\ideal\mathbb{Z}$.

De manière générale, $I$ est un idéal de $\mathbb{Z}$ si et seulement si $I=n\mathbb{Z}$ pour un $n\in\mathbb{Z}$.
\end{ex}

\begin{ex}
$\langle x,2\rangle$ est un idéal de $\mathbb{Z}[x]$ (les polynomes avec coeficients dans $\mathbb{Z}$. Montrons que $\langle x,2\rangle$ n'est pas principal.

Si $\langle x,2\rangle$ est principal alors il existe $f(x)\in\mathbb{Z}[x]$ tel que $\langle f(x)\rangle=\langle x,2\rangle$. Donc $2=g(x)f(x)$ pour un certain $g(x)\in\mathbb{Z}[x]$. Cependant le degré du produit de deux polynomes  est toujours plus grand ou égal à leurs degrés respectif. Donc $g(x)=c_g$ et $f(x)=c_f$ et $c_g,c_f\in\left\{\pm1,\pm2\right\}$. De plus, $f(x)|x$ donc $f(x)=\pm1$. Ce qui veut dire que $1\in\langle f \rangle=\langle x,2\rangle(-1)(-1)=1$. 

Il existe alors $h(x),k(x)\in\mathbb{Z}[x]$ tel que $1=h(x)x+k(x)2$ parcontre $1$ est impair, $h(x)x$ est pair et $k(x)2$ est pair. La somme de deux termes pairs ne peuvent pas donner un nombre impair. Par contradiction $\langle x,2\rangle$ n'est pas un idéal principal.
\end{ex}

\begin{ex}
Soit $A=M_2\left(\mathbb{R}\right)$ et considérons $I=\left\{ \left( 
\begin{array}{cc}
0&b\\
0&d
\end{array}
\right) | b,d\in\mathbb{R}\right\}$. $I$ est un idéal à gauche mais pas à droite.

Considérons $J=\left\{ \left( 
\begin{array}{cc}
a&b\\
0&0
\end{array}
\right) | a,b\in\mathbb{R}\right\}$. $J$ est un idéal à droite mais pas à gauche.
\end{ex}

\section{Idéaux Premiers, Idéaux Maximaux et Opérations sur les idéaux}
\begin{defn}
Un idéal propre $P$ d'un anneau commutatif $A$ est dit premier si pour tout $x,y\in A$ tel que $xy\in P$ alors $x\in P$ ou $y\in P$.
\end{defn}

\begin{ex}
Soit $n\in\mathbb{Z}^+$ alors $n\mathbb{Z}$ est premier si et seulement si $n$ est un nombre premier.
\end{ex}
\begin{proof}
$n\mathbb{Z}$ est premier donc $ab\in n\mathbb{Z}\Longrightarrow a\in n\mathbb{Z}$ ou $b\in n\mathbb{Z}$. Si $n$ est premier alors $n|ab\Longrightarrow n|a$ ou $n|b$.

Supposons que $n$ est composé. Il existe donc $a,b\in\mathbb{Z}\setminus\{0,1\}$ tel que $n=ab$. Alors $|n|>|a|$ et $|n|>|b|$. Il s'en suit que $n\not|a$ et $n\not|b$. Donc $ab\not\in n\mathbb{Z}$ donc $n\mathbb{Z}$ n'est pas premier. Par contradiction $n$ est premier.
\end{proof}

\begin{defn}
Soit $A$ un anneau, soit $I$ et $J$ des idéaux de $A$ --- gauche, droite ou bilataire. On définit les opérations suivantes.

\begin{align*}
I+J:=\left\{i+j | i\in I, j\in J\right\}
\end{align*}

\begin{align*}
IJ:=\left\{\sum\limits_{k=1}^N i_kj_k | i_k\in I, j_k\in J, \forall N\in\mathbb{N}\right\}
\end{align*}

$I+J$, $IJ$ et $I\cap J$ sont des idéaux.
\end{defn}

%\begin{thm}
%Soit $P$ un idéal propre d'un anneau  commutatif.
%\end{thm}

\end{document}
